\documentclass[10pt]{article}

\usepackage{amsmath, amssymb, amsthm}
\usepackage[usenames, dvipsnames]{color}
\usepackage[boxed, linesnumbered]{algorithm2e}	
\usepackage[paper=a4paper, dvips, top=2.5cm, left=2.5cm, right=2.5cm, foot=1.5cm, bottom=3.0cm]{geometry}
\usepackage[T1]{fontenc}
\usepackage{charter}
\usepackage{hyperref}
\hypersetup{colorlinks, citecolor=black, filecolor=black, linkcolor=black, urlcolor=black}

\newtheorem{theorem}{Theorem}
\newtheorem{corollary}[theorem]{Corollary}
\newtheorem{lemma}[theorem]{Lemma}

\begin{document}

\section*{Edmonds-Karp algorithm}
To improve the $O(mC)$ bound of the Ford-Fulkerson algorithm, we can be careful while choosing augmenting paths. The Edmonds-Karp algorithm chooses a \emph{shortest} augmenting $s-t$ path where each edge has unit distance, ie an $s-t$ path with the fewest number of edges.

For the purpose of analysis, let $d_f[v]$ be the shortest distance(number of edges) of the vertex $v$ from source $s$ in the residual graph $G_f$.

\begin{lemma}
During the course of the algorithm, $d_f[\cdot]$ value of any vertex does not decrease.
\end{lemma}
\begin{proof}
We use proof by contradiction. Let $G_f$ be the first residual graph in which the shortest distance from $s$ of some vertex decreased and let $v$ be such a vertex in $G_f$ which is closest to $s$, ie the one with minimum $d_f[v]$. Let $G_{f'}$ be the flow network just before $G_f$. Then, 
\begin{equation}
\label{eq:comparev}
d_f[v] < d_{f'}[v]
\end{equation}
Let $s \leadsto u \rightarrow v$ be the shortest $s-v$ path in $G_f$. Then,
\begin{equation}
\label{eq:distfv}
d_f[v] = d_f[u] + 1
\end{equation}
Because $v$ was the closest vertex to $s$ in $G_f$ whose shortest distance decreased, $d_f[u]$ did not decrease, ie,
\begin{equation}
\label{eq:compareu}
d_f[u] \geq d_{f'}[u]
\end{equation}
Now, $(u,v) \notin E_{f'}$. Because if $(u,v)$ was an edge in $G_{f'}$,
\begin{align*}
d_{f'}[v]   & \leq d_{f'}[u] + 1    & \\
            & \leq d_f[u] + 1       & \text{(from equation \ref{eq:compareu})} \\ 
            & = d_f[v]              & \text{(from equation \ref{eq:distfv})} \\
\end{align*}
But this contradicts our assumption made in equation \ref{eq:comparev}. Thus $(u,v) \notin E_{f'}$. But since $(u,v) \in E_f$, it must have been that in $G_{f'}$, our algorithm chose an augmenting path that contained the edge $(v,u)$. Since we always choose shortest paths, 
\begin{align*}
d_{f'}[u]   & = d_{f'}[v] + 1    & \\
d_{f'}[v]   & = d_{f'}[u] - 1    & \\
            & \leq d_f[u] - 1    & \text{(from equation \ref{eq:compareu})} \\ 
            & = d_f[v] - 2       & \text{(from equation \ref{eq:distfv})} \\ 
d_{f'}[v]   & < d_f[v]          & \\
\end{align*}
This contradicts our initial assumption(equation \ref{eq:comparev}). Thus such a vertex $v$ cannot exist.
\end{proof}
\bigskip

\begin{lemma}
Let $(u,v)$ be the bottleneck edge in $G_f$. Then the next time $(u,v)$ becomes a bottleneck edge, $d[u]$ increases by at least 2.
\end{lemma}
\begin{proof}
Since $(u,v)$ is chosen as the bottleneck edge in $G_f$ and we choose shortest paths,
\begin{equation}
\label{eq:distfuv}
d_f[v] = d_f[u] + 1
\end{equation}
In all subsequent residual graphs, the edge $(u,v)$ will not be present unless the algorithm augments flow along the edge $(v,u)$. Let $G_{f'}$ be the first such residual graph. Then,
\begin{equation}
\label{eq:distf'vu}
d_{f'}[u] = d_{f'}[v] + 1
\end{equation}
By the previous lemma, we have,
\begin{equation}
\label{eq:compareff'v}
d_{f'}[v] \geq d_f[v]
\end{equation}
Using equation \ref{eq:compareff'v} in equation \ref{eq:distf'vu} we get,
\begin{align*}
d_{f'}[u]   & \geq d_f[v] + 1       & \\
            & = d_f[u] + 2          & \text{(from equation \ref{eq:distfuv})} \\
\end{align*}
\end{proof}
\bigskip

\begin{theorem}
Edmonds-Karp algorithm takes $O(m^2n)$ time to compute max-flow in a network.
\end{theorem}
\begin{proof}
Initially the distance of a node $u$ from $s$ is at least $0$ and if distance of $u$ from $s$ becomes more than $n$, it is unreachable from $s$(since we always choose a simple path for augmentation). Also, between each time the edge $(u,v)$ becomes a bottleneck edge, $d[u]$ increases by at least $2$. Thus each edge can become bottleneck edge at most $n/2$ times. Since there are $O(m)$ edges, and each augmenting path has at least one bottleneck edge, it follows that the total number of augmenting paths is at most $O(mn)$. To find one augmenting path, it takes $O(m)$ time using breadth-first search. Thus the total running time is $O(m^2n)$.
\end{proof}


\end{document}


